\documentclass[a4paper,12pt]{article}

\usepackage[slovene]{babel}
\usepackage{amsfonts,amssymb,amsmath}
\usepackage[utf8]{inputenc}
\usepackage[T1]{fontenc}
\usepackage{lmodern}
\usepackage{graphicx}
\usepackage{amsmath} %%%dodan
\usepackage{url}
\usepackage{float} 



%%\def\qed{$\hfill\Box$} % konec dokaza
%\def\qedm{$\qquad\Box$} % konec dokaza v matematičnem načinu
%\newtheorem{izrek}{Izrek}
%\newtheorem{trditev}{Trditev}
%\newtheorem{posledica}{Posledica}
%\newtheorem{lema}{Lema}
%\newtheorem{opomba}{Opomba}
%\newtheorem{definicija}{Definicija}
%\newtheorem{zgled}{Zgled}
\renewcommand{\thesubsection}{\thesection.\alph{subsection}}

\author{Mateja Možina \\ Fakulteta za matematiko in fiziko}
\title{
\huge  {Projektna naloga za Statistiko} }

\begin{document}


%%%%%%%%%%%%%%%%%%%%%%%%%%%%%%%%%%%%%%%%%%%%%%%%%%%%%%%%%%%%%%%%%%%%%


\maketitle

%%%%%%%%%%%%%%%%%%%%%%%%%%%%%%%%%%%%%%%%%%%%%%%%%%%%%%%%%%%%%%%%%%%%%

\newpage

\section{Prva naloga}
V mestu Kibergrad živi 43.886 družin, razporejenih v 4 četrti. Zanima nas povprečno število otrok. Nalogo rešujemo v 
programu \emph{Matlab} z datoteko \textbf{prva\_naloga.m}.
%
%
\subsection{Povprečno število otrok z enostavnim slučajnim vzorčenjem}
Izmed $N = 43.886$ naključno izberemo $n = 400$ družin in z
\begin{equation*}
    \overline{X} = \frac{X_1+X_2 + \dots + X_{400}}{400},
\end{equation*}
kjer je 
\begin{center}
    $X_i=$ število otrok v $i$-ti družini,
\end{center}
izračunamo, da je povprečno število otrok enako $0.955$. Standardno napako dobimo z
\begin{equation*}
    \widehat{SE_{+}} = \sqrt{\frac{N-n}{n-1}\frac{1}{N}\frac{1}{n}\sum_{i=1}^{n}(X_i - \overline{X})^2}
\end{equation*}
in tako je $\widehat{SE_{+}} \doteq 0.0557$.\\
Za interval zaupanja pri $\alpha=0.05$ pa vemo, da je
\begin{equation*}
    \overline{X} - F^{-1}_{Student(n-1)}\left(1-\frac{\alpha}{2}\right)\widehat{SE_{+}} < \mu < \overline{X} + F^{-1}_{Student(n-1)}\left(1-\frac{\alpha}{2}\right)\widehat{SE_{+}}.
\end{equation*}
Z našimi podatki bo torej
\begin{equation*}
    0.8455 < \mu < 1.0644.
\end{equation*}
%
%
\subsection{Povprečno število otrok s stratificiranim vzorčenjem glede na četrt}
Pri stratificiranem vzorčenju razdelimo populacijo velikosti $N$ na $k$ stratumov velikosti $N_1, \dots, N_k$. V našem primeru bomo imeli 4 
stratume - četrti v katerih stanuje družina.  Velja, da je 
$N_1 + N_2 + N_3 + N_4 = N$, z $W_i = \frac{N_i}{N}$ pa označimo še velikostne deleže. Pri stratificiranem vzorčenju s proporcionalno alokacijo 
iz vsakega stratuma izberemo naključni vzorec $n_i$, tako da velja
\begin{equation*}
    \frac{n_1}{N_1} = \frac{n_2}{N_2} = \frac{n_3}{N_3} = \frac{n_4}{N_4}.
\end{equation*}
Naš vzorec je enak $n=400$, torej rabi še veljati
\begin{equation*}
    n_1 + n_2 + n_3 + n_4 = 400.
\end{equation*}
Iz teh pogojev dobimo, da je 
\begin{equation*}
    n_1\left(1 + \frac{10390}{10149} + \frac{13457}{10149} + \frac{9890}{10149}\right) = 400
\end{equation*}
oziroma, da je
\begin{equation*}
    n_1 = \frac{400}{\left(1 + \frac{10390}{10149} + \frac{13457}{10149} + \frac{9890}{10149}\right)}.
\end{equation*}
Tako dobimo, da so
\begin{equation*}
    n_1 = 92, \; n_2 = 95, \; n_3 = 123 \; \; \text{in} \; \; n_4 = 90,
\end{equation*}
kjer so vsa števila zaokrožena, razen $n_1$ je zaokrožen navzdol.\\
Sedaj iz vsakega stratuma izberemo naključni vzorec velikosti $n_i$ in s formulo
\begin{equation*}
    \overline{X_{i}} = \frac{1}{n_i}\sum_{j=1}^{n_i}X_{ij}
\end{equation*}
za vsak stratum izračunamo povprečno število otrok. Tako so
\begin{align*}
    \overline{X_{1}} &= \frac{1}{92}\sum_{j=1}^{92}X_{1j} = 0.837,\\
    \overline{X_{2}} &= \frac{1}{95}\sum_{j=1}^{95}X_{2j} = 1.095,\\
    \overline{X_{3}} &= \frac{1}{123}\sum_{j=1}^{123}X_{3j} = 1.033,\\
    \overline{X_{4}} &= \frac{1}{90}\sum_{j=1}^{90}X_{4j} = 1.144.
\end{align*}
Tako bo povprečno število otrok
\begin{equation*}
    \overline{X} = \sum_{i=1}^{4}W_{i}\overline{X_{i}} = 1.027.
\end{equation*}
Da bi ocenili standardno napako, rabimo najprej izračunati variance posameznih stratumov. Za to uporabimo
\begin{equation*}
    \widehat{\sigma_i}^{2} = \frac{1}{n_i-1}\sum_{j=1}^{n_i}(X_{ij} - \overline{X_{i}})^2, 
\end{equation*}
kar nam da
\begin{equation*}
    \widehat{\sigma_1}^{2} = 1.171, \; \widehat{\sigma_2}^{2} = 1.619, \; \widehat{\sigma_3}^{2} = 1.425, \; \widehat{\sigma_4}^{2} = 1.788.
\end{equation*}
Standardno napako izračunamo s formulo
\begin{equation*}
    \widehat{SE} = \sqrt{\sum_{i=1}^{4}W_i^2\frac{1}{n_i}\widehat{\sigma_i}^{2}},
\end{equation*}
kar nam da
\begin{equation*}
    \widehat{SE} = 0.0611.
\end{equation*}
Izračunajmo še interval zaupanja pri $\alpha=0.05$. Vemo, da je 
\begin{equation*}
    T = \frac{\overline{X}- \mu}{\widehat{SE}} \mathrel{\dot\sim} Student(\nu),
\end{equation*}
kjer je 
\begin{equation*}
    \nu = \frac{SE^4}{\sum_{i=1}^{4}\frac{W_i^4\sigma_i^4}{n_i^2(n_i -1)}}.
\end{equation*}
\textbf{Opomba: Ta enakost je dobljena iz spletne učilnice Statistika - Gradiva - Stratificirani modeli, čisto na koncu.}

Sedaj namesto standardne napake in varianc, v enačbo damo njihove cenilke. Tako bo
\begin{equation*}
    \widehat{\nu} = 387.742
\end{equation*}
in aproksimirani interval zaupanja pri stopnji tveganja $\alpha=0.05$ je
\begin{equation*}
    \overline{X} - F^{-1}_{Student(\widehat{\nu})}\left(1-\frac{\alpha}{2}\right)\widehat{SE} < \mu < \overline{X} + F^{-1}_{Student(\widehat{\nu})}\left(1-\frac{\alpha}{2}\right)\widehat{SE},
\end{equation*}
kar bo v našem primeru
\begin{equation*}
    0.907 < \mu < 1.147.
\end{equation*}
Vidimo, da nam stratificirano vzorčenje s proporcionalno alokacijo da večje povprečno število otrok, prav tako pa tudi večjo standardno napako. Interval zaupanja je pri
enostavnem slučajnem vzorčenju manjši (pri enostavnem slučajnem vzorčenju je dolžine $0.21$, pri stratificiranem s proporcionalno alokacijo
pa približno $0.24$), vendar ne za veliko.
%
%
%
%
\section{Druga naloga}
Podane imamo podatke o skokih, ki jih ptiči naredijo med dvema letoma, kjer je število skokov vedno vsaj $1$, saj štejemo tudi zadnji skok, ko 
poleti. Nalogo rešujemo s programom \textbf{druga\_naloga.m}.
\subsection{Geometrijska porazdelitev}
Označimo $X_i =$ število skokov pri $i$-tem opažanju, iščemo pa geometrijsko porazdelitev, ki se najbolje prilega našim podatkom, 
zato predpostavimo, da so $X_i \sim Geom(p)$ in med sabo neodvisne. Torej rabimo najti cenilko za $p$ za 
\begin{equation*}
    P(X = k) = pq^{k-1},
\end{equation*}
kjer je $q = 1-p$. Vseh opažanj je $n = 130$, število vseh skokov pa $363$. 
Besedna zveza 
">se najbolje prilega"< nam pove, da iščemo cenilko po metodi največjega verjetja.
Verjetje bo torej
\begin{align*}
    L(p|X_1, \ldots, X_n) &= \prod_{i=1}^{n}P(X_i = x_i) = \prod_{i=1}^{n}(1-p)^{x_i-1}p = (1-p)^{\sum_{i=1}^{n}x_i-n}p^n \\
    &= (1-p)^{S-n}p^n,
\end{align*}
kjer je $S = \sum_{i=1}^{n}x_i$ število vseh skokov.
Če logaritmiramo, dobimo
\begin{align*}
    l(p|X_1, \ldots, X_n) = \ln((1-p)^{S-n}p^n)= (S-n)\ln(1-p) +n\ln(p),
\end{align*}
kar sedaj odvajamo po $p$
\begin{equation*}
    \frac{\partial l(p|X_1, \ldots, X_n)}{\partial p} = \frac{S-n}{p-1} + \frac{n}{p}.
\end{equation*}
Enačimo to z $0$ in dobimo 
\begin{equation*}
    \widehat{p} = \frac{n}{S},
\end{equation*}
kar je naša cenilka za $p$. Pri naših podatkih bo znašala $\widehat{p} =0.3581$.
%
%
\subsection{Grafični prikaz}
Poglejmo kako dobro se ta geometrijska porazdelitev ujema z našimi podatki. Vemo, da je verjetnost, da ptič poskoči $k$-krat enaka
\begin{equation*}
    P(X = k) = \widehat{p}q^{k-1}
\end{equation*}
za $\widehat{p} =0.3581$, torej bomo v $n$-tih opažanjih videli 
\begin{equation*}
    n \cdot P(X = k) = n \cdot \widehat{p}(1-\widehat{p})^{k-1}
\end{equation*}
opažanj oziroma frekvenc, da so ptiči poskočili $k$-krat.
\begin{figure}[H]  
	\centering
	\includegraphics[scale=0.7]{slika1.png}
	\caption{Graf porazdelitev.}
\end{figure}
Vidimo, da se naši podatki in geometrijska porazdelitev dobro ujemata, največje odstopanje pa je približno $3.3$.
%
%
\subsection{En presledek}
Recimo, da smo opazili samo en presledek med letoma. Naša cenilka v tem primeru ne bo nepristranska (za $n=1$), saj velja
\begin{equation*}
    E(\widehat{p}) = E\left(\frac{1}{X_1}\right) = \sum_{k=1}^{\infty}\frac{1}{k}p(1-p)^{k-1} = 
    p + \underbrace{\sum_{k=2}^{\infty}\frac{1}{k}p(1-p)^{k-1}}_\text{pozitivne vrednosti} > p.
\end{equation*}
Konstruirajmo sedaj nepristransko cenilko. Iščemo torej funkcijo $f$, da bo veljalo
\begin{equation*}
    E(f(X)) = \sum_{k=1}^{\infty}f(k)pq^{k-1} = p
\end{equation*}
oziroma
\begin{equation*}
    \sum_{k=1}^{\infty}f(k)q^{k-1} = 1.
\end{equation*}
To nam reši funkcija
\begin{equation*}
    f(k) = \begin{cases}
        1  &  k =1, \\
        0 & k >1,
    \end{cases}
\end{equation*}
kar je naša iskana nepristranska cenilka za $p$ (je tudi edina možna).
%
%
\subsection{Več presledekov}
Tudi, če opazimo več presledkov $X, \ldots, X_n$, je cenilka $\widehat{p}$  še vedno pristranska. Vemo, da je vsota $n$ geometrijskih porazdelitev 
$X_i \sim Geom(p)$ porazdeljena negativno binomsko, torej $Z = \sum_{i=1}^{n}X_i \sim NegBin(n,p)$. Tako je
\begin{align*}
    E(\widehat{p}) &= nE\left(\frac{1}{Z}\right)= \sum_{k=n}^{\infty}\frac{n}{k}\binom{k-1}{n-1}p^nq^{k-n}\\
    &= p\sum_{k=n}^{\infty}\frac{n}{k}\frac{k-1}{n-1}\binom{(k-1)-1}{(n-1)-1}p^{n-1}q^{(k-1)-(n-1)} \\
    &> p \cdot \underbrace{\sum_{k=n}^{\infty}\binom{(k-1)-1}{(n-1)-1}p^{n-1}q^{(k-1)-(n-1)}}_\text{=1, ker je to vsota vseh verjetnosti v NegBin(n-1,p)} = p
\end{align*}
in cenilka za $p$ je pristranska.

Poglejmo sedaj cenilko 
$$\check{p}=\frac{n-1}{S-1},$$
kjer je $S$ skupno ševilo opaženih skokov.
Ko izračunamo pričakovano vrednost
\begin{align*}
    E(\check{p}) &=  \sum_{k=n}^{\infty}\frac{n-1}{k-1}\binom{k-1}{n-1}p^nq^{k-n} \\
    &= \sum_{k=n}^{\infty}\binom{k-2}{n-2}p^nq^{k-n} \\
    &= p \cdot \underbrace{\sum_{k=n}^{\infty}\binom{(k-1)-1}{(n-1)-1}p^{n-1}q^{(k-1)-(n-1)}}_\text{=1, ker je to vsota vseh verjetnosti v NegBin(n-1,p)} \\
    &= p,
\end{align*}
vidimo, da je ta cenilka nepristranska. Za naš primer bomo dobili, da je $\check{p}=0.3564$, kar se od pristranske $\widehat{p}$ razlikuje 
šele na tretji decimalki.
%
%
%
%
\section{Tretja naloga}
Podane imamo podatke o težah piščancev, ki so bili hranjeni z različnimi dietami v roku treh tednov. Nalogo rešujemo s 
programom \textbf{tretja\_naloga.m}.
\subsection{Pričakovana teža v enem dnevu}
Z enostavno linearno regresijo ocenimo koliko teže pridobi piščanec v enem dnevu. Na voljo imamo podatke o petdesetih piščancih, hranjenih 
v treh tednih s štirimi različnimi dietami, vseh podatkov pa je $n = 578$. Naj bo 
\begin{equation*}
    Y = \begin{bmatrix}
        y_1 \\
        y_2\\
        \vdots \\
        y_{579}
        \end{bmatrix}
\end{equation*}
vektor vseh podanih tež in 
\begin{equation*}
    X = \begin{bmatrix}
        1 & x_1 \\
        1 & x_2\\
        \vdots & \vdots \\
        1 & x_{579}
        \end{bmatrix}
\end{equation*}
matrika, kjer so $x_i$ dnevi. Tako bo naša cenilka za $\widehat{\beta}$
\begin{equation*}
    \widehat{\beta}= \left(X^{T}X\right)^{-1}X^TY
    = \begin{bmatrix}
        27.467  \\
        8.803 
        \end{bmatrix}.
\end{equation*}
Torej piščanec v enem dnevu pridobi približno $8.803$ gramov. Poglejmo si to še na grafu.
\begin{figure}[H]  
	\centering
	\includegraphics[scale=0.7]{slika2.png}
	\caption{Premica naraščanja teže.}
\end{figure}
Vidimo, da so na začetku piščanci imeli zelo podobne teže - malo pod $50$~g, po enaindvajstih dnevih pa so razlike velike - najmanjša 
izmerjena teža je bila okoli $60$ g, največja pa okoli $370$ g. Glede na sliko se zdi, da se premica dobro prilega našim podatkom.

\textbf{Opomba: Tukaj manjkajo še napake $\epsilon_i$. Iz podatkov, ki jih imamo na grafu, naj bi ti epsiloni bili oblike 
$\epsilon_i =\sigma^2 x_i^2 + \tau^2$, kjer sta sigma in tau neznani. Splošen model - $E(\epsilon_i)=0$ in $Var(\epsilon)= \sigma^2I$ -
tako ne velja, velja samo prva enakost.}
%
%
\subsection{Vpliv diete na težo}
Poglejmo, ali dieta vpliva na težo piščanca. Naša ničelna hipoteza $H_0$ je, da dieta ne vpliva na težo, alternativna domneva $H_1$ pa
je, da dieta vpliva na težo. Naj bo $\beta_0$ maša piščanca na začetku, $\beta_i, i = 1,2,3,4$ pa koliko piščanec pridobi teže v enem dnevu glede 
na $i$-to dieto. Sestavimo matriko $X$
$$
    X = \begin{bmatrix}
            1 & x_{1,1} & 0 & 0 & 0 \\
            \vdots & \vdots & \vdots & \vdots & \vdots \\
            1 & x_{1,l_1} & 0 & 0 & 0 \\
            1 & 0 & x_{2,1} & 0 & 0 \\
            \vdots & \vdots & \vdots & \vdots & \vdots \\
            1 & 0 & x_{2,l_2} & 0 & 0 \\
            1 & 0 & 0 & x_{3,1} & 0 \\
            \vdots & \vdots & \vdots & \vdots & \vdots \\
            1 & 0 & 0 & x_{3,l_3} & 0 \\
            1 & 0 & 0 & 0 & x_{4,1} \\
            \vdots & \vdots & \vdots & \vdots & \vdots \\
            1 & 0 & 0 & 0 & x_{4,l_4}
        \end{bmatrix},
$$ 
ki ima v prvem stolpcu same enice, v drugem podatke o dnevih za piščance hranjene s prvo dieto, v tretjem stolpcu o dnevih za piščance hranjene
z drugo dieto, v četrtem o dnevih za piščance hranjene s tretjo dieto in v petem s četrto dieto. Vektor $\beta$ je tako
$$\beta = \begin{bmatrix}
    \beta_0 \\
    \beta_1 \\
    \beta_2 \\
    \beta_3 \\
    \beta_4 
\end{bmatrix},$$
kjer je $\beta_0$ začetna teža, $\beta_i$ pa pričakovana pridobitev teže v enem dnevu z dieto $i$. Torej, naša ničelna hipoteza je, da
$$ H_0 : \beta_1 = \beta_2 = \beta_3 = \beta_4
$$
oziroma, da je 
$$\beta = \beta_0 \begin{bmatrix}
    1 \\
    0 \\
    0 \\
    0 \\
    0 
\end{bmatrix} + 
\beta_1 \begin{bmatrix}
    0 \\
    1 \\
    1 \\
    1 \\
    1 
\end{bmatrix}.
$$
Dimenzija podprostora $W$, ki ga razpenjata ta dva vektorja je torej $q$ = dim$W$ = $2$, dimenzija prostora $V$ = Im(X) pa $p$ = dim$V$ = $5$.
Naj bo $H$ ortogonalni projektor na $V$ in $K$ ortogonalni projektor na $W$. $H$ izračunamo z 
$$ H = X(X^{T}X)^{-1}X^{T},
$$
$K$ pa podobno, samo da namesto $X$-a vstavimo
$$A =X\begin{bmatrix}
    1 &0\\
    0 &1\\
    0 &1\\
    0 &1\\
    0 &1 
\end{bmatrix}.$$
Preizkusna statistika bo
$$F = \frac{\frac{\|(H - K)Y\|^2}{p - q}}{\frac{\|(I_n - H)Y\|^2}{n - p}},$$
domnevo $H_0$ pa bomo zavrnili pri stopnjah tveganja $\alpha=0.05$ in $\alpha=0.01$, če je 
$$ F \geqslant F^{-1}_{Fisher(p-q, n-p)}\left(1-\alpha\right).
$$

\textbf{Opomba: Ta enakost je dobljena iz spletne učilnice Statistika - Gradiva - Statistično sklepanje pri linearni regresiji.}

Za naše podatke dobimo, da je
\begin{align*}
    &F = 59.32, \\
    &F^{-1}_{Fisher(p-q, n-p)}\left(1-0.05\right) = 2.62, \\
    &F^{-1}_{Fisher(p-q, n-p)}\left(1-0.01\right) = 3.82,
\end{align*}
torej v obeh primerih hipotezo $H_0$ zavrnemo, torej izbira diete vpliva na težo piščanca. Podobno kot v primeru a lahko izračunamo $\beta$
$$\beta = \begin{bmatrix}
    27.86 \\
    7.05 \\
    8.66 \\
    10.79 \\
    9.91 
\end{bmatrix}$$
in vidimo da je bila dieta 3 najbolj učinkovita. Še grafično
\begin{figure}[H]  
	\centering
	\includegraphics[scale=0.7]{slika3.png}
	\caption{Različne diete.}
\end{figure}
\end{document}